\documentclass[11pt,chapterprefix=true,toc=bibliography,numbers=noendperiod,
               footnotes=multiple]{scrbook}
\usepackage{fixltx2e} % LaTeX patches, \textsubscript
\usepackage{microtype}
\usepackage{cmap} % fix search and cut-and-paste in Acrobat
\usepackage{ifthen}
\usepackage[oldstylenums,largesmallcaps]{kpfonts}
\usepackage[T1]{fontenc}
\usepackage[utf8]{inputenc}
\usepackage[british]{babel}
\usepackage{csquotes}
\usepackage{tabularx}
\usepackage{booktabs}
\usepackage{algorithmicx}
\usepackage{algpseudocode}
\usepackage{algorithm}
\usepackage{listings}
\usepackage[table,hyperref,dvipsnames]{xcolor}
\usepackage[hidelinks]{hyperref}
\usepackage[autocite=footnote,citestyle=authoryear-comp,bibstyle=authoryear,
            isbn=false,doi=false,backend=biber]{biblatex}
\usepackage{nag}

%%% Custom LaTeX preamble
% serif, non-bold headings:
\addtokomafont{chapter}{\mdseries}
\addtokomafont{disposition}{\rmfamily}
\addtokomafont{descriptionlabel}{\rmfamily}
\addtokomafont{pageheadfoot}{\itshape}
% section numbering up to subsection
\setcounter{secnumdepth}{2}
% hyperlinks
\urlstyle{same} % normal text font (alternatives: tt, rm, sf)
\hypersetup{
  pdftitle={Extending Raft with structured voting},
  pdfauthor={Leonhard Markert (lm510), Emmanuel College}
}
\addbibresource{Bibliography.bib}
\pagestyle{headings}

%%% Body
\begin{document}

\frontmatter

\begin{titlepage}

\rightline{\large\textit{Leonhard Markert}} \medskip
\rightline{\large\textit{Emmanuel College}} \medskip
\rightline{\large\textit{lm510}}

\vfil

\centerline{\large Computer Science Part \textsc{ii} Project} \vspace{0.4in}
\centerline{\Large\textbf{Extending Raft with structured voting}} \vspace{0.3in}
\centerline{\large\textit{\today}}

\vfil

\begin{center}
{\renewcommand{\arraystretch}{2}%
\begin{tabularx}{316pt}{rX}
\textbf{Project Supervisors} & \textit{Malte Schwarzkopf} and \textit{Ionel Gog} \\
\textbf{Director of Studies} & \textit{Dr Jonathan Hayman} \\
\textbf{Overseers} & \textit{Dr Markus Kuhn} and \textit{Dr Neal Lathia}
\end{tabularx}}
\end{center}

\end{titlepage}

\chapter*{Proforma\label{ch:proforma}}

\begin{center}
{\renewcommand{\arraystretch}{1.5}%
\begin{tabularx}{330pt}{rX}
\textbf{Name and College} & Leonhard Markert, Emmanuel College \\
\textbf{Project Title} & Extending Raft with structured voting \\
\textbf{Examination} & Computer Science Tripos, Part \textsc{ii}, June 2014 \\
\textbf{Word Count} & XXX words \\
\textbf{Project Originator} & Leonhard Markert \\
\textbf{Project Supervisors} & Malte Schwarzkopf and Ionel Gog
\end{tabularx}}
\end{center}

\section*{Original aims of the project\label{sc:original-aims}}

XXX

\section*{Work completed\label{sc:work-completed}}

XXX

\section*{Special Difficulties\label{sc:special-difficulties}}

None.

\section*{Declaration of Originality\label{sc:declaration-of-originality}}

I, Leonhard Markert of Emmanuel College, being a candidate for Part~\textsc{ii} of the Computer Science Tripos, hereby declare that this dissertation and the work described in it are my own work, unaided except as may be specified below, and that the dissertation does not contain material that has already been used to any substantial extent for a comparable purpose.

\vspace{0.3in}
Signed

\vspace{0.2in}
Date \hspace{0.4in} \today

\chapter*{Acknowledgements\label{ch:acknowledgements}}

\begin{itemize}
    \item Malte Schwarzkopf and Ionel Gog
    \item Christian Storm
    \item Andrew Stone
\end{itemize}

\tableofcontents

\mainmatter

\chapter{Introduction\label{ch:introduction}}

% \begin{itemize}
    % \item Principal motivation for the project
    % \item How the work fits into the broad area of surrounding CS
    % \item Survey of related work
% \end{itemize}

% \begin{itemize}
    % \item Raft: consensus protocol, asymmetric: leader, follower, candidate -- elections, majority voting by default; state machine, backend, ``understandable'' Paxos replacement -- strongly consistent (all operations are seen in the same order by all nodes); correctness proof \parencite{raft} \parencite{raftproof}
    % \item Structured voting: more efficient / available (less cost), example: Grid protocol, key insight: don't need majority to guarantee mutual exclusion; tree-shaped voting structures: generalised framework; logical structure on nodes; smaller quorums
    % \item Motivation: combine Raft and structured voting (\textit{write} quorums specifically)
    % \item Failure model: no Byzantine failures -- servers either work or not; fail-stop, permanent/volatile memory, lost/delayed messages but not corrupted
% \end{itemize}

\section{Motivation\label{sc:motivation}}

As ever-increasing amounts of data are being handled in commercial settings as well as in research, distributed systems for processing and storage are becoming more and more ubiquitous. One main driver of this development is the need to increase fault tolerance.

According to Eric Brewer's famous \textsc{cap} theorem \autocite{cap}, a partition tolerant system cannot be both strictly consistent and maximally available.\footnote{The terms consistency, availability and partition tolerance are defined in \autoref{ssc:cap-acid-and-base}.} Many recent distributed data stores sacrifice consistency for availability. Some applications, however, require strong consistency guarantees -- data backup and configuration management systems, for example. Consensus protocols like Paxos \autocite{paxos} and Raft \autocite{raft} guarantee consistency at the cost of decreased availability.

In this project, I added support for structured voting schemes to an existing implementation of Raft. My aim was increase the availability and scalability of data storage systems built on top of this implementation while still providing the same consistency guarantees as the original Raft algorithm.

Increasing the availability of a consistent distributed system would have two effects:

\begin{itemize}
    \item Applications that require consistency and already use a consensus algorithm that provides consistency could have their availability increased by migrating to this new implementation;
    \item Applications that require high availability and currently run an algorithm that does not guarantee consistency, might switch to this new implementation too.
\end{itemize}

\section{Challenges\label{sc:challenges}}

\section{Related work\label{sc:related-work}}

\subsection{Distributed systems: \textsc{cap}, \textsc{acid} and \textsc{base}\label{ssc:cap-acid-and-base}}

Eric Brewer's famous \textsc{cap} theorem states that any distributed system can provide at most two out of the three desirable properties consistency (\textsc{c}), availability (\textsc{a}), and partition tolerance (\textsc{p}).\autocite{cap} The following definitions are adapted from \textcite{capproof}:

\begin{description}
    \item[Consistency] There must exist a total order on all operations such that each operation looks as if it were completed at a single instant. An important consequence of this \emph{linearisable} (or \emph{atomic}) consistency guarantee is that any read operation that begins after a write operation completes must return that value, or the result of a later write operation.\footnote{As \textcite{capproof} point out, the term \emph{consistency} is highly overloaded. Note that the above notion of atomic consistency subsumes what is called atomicity and consistency in the context of \textsc{acid} (\enquote{atomic, consistent, isolated, durable}) databases.}
    \item[Availability] Every request received by a non-failing node in the system must result in a response.
    \item[Partition tolerance] The system continues to operate despite arbitrary message loss. This includes network partitions, where all messages sent from nodes in one component of the partition to nodes in another component are lost.
\end{description}

Out of those three, partition tolerance is required in almost all cases. As \citeauthor{needp} puts it in his article \citetitle{needp}:

\begin{quote}
    For a distributed \dots{} system to \emph{not} require partition tolerance it would have to run on a network which is guaranteed to never drop messages \dots{} and whose nodes are guaranteed to never [fail]. [These] types of systems \dots{} don't exist.
\end{quote}

This leaves designers of distributed systems with the task of finding the right trade-off between consistency and availability, both of which should be considered as continuous rather than binary properties \autocite{cap12}. With the rise of commercial databases of unprecedented scale over the last decade, consistency (in its strict form as defined above) is in many cases sacrificed for increased availability.%
\footnote{This has been heralded as a paradigm shift from the \textsc{acid} to the \textsc{base} (\enquote{basically available, soft state, eventually consistent}) model \parencite{base}.} %
The authors of \citetitle{dynamo} describe this as follows:%
\footnote{Notable other eventually consistent systems include Cassandra \parentext{\citeurl{cassandra}}, Riak \parentext{\citeurl{riak}} and HBase \parentext{\citeurl{hbase}}.}

\begin{quote}
    For systems prone to server and network failures, availability can be increased by using optimistic replication techniques, where changes are allowed to propagate to replicas in the background, and concurrent, disconnected work is tolerated. The challenge with this approach is that it can lead to \emph{conflicting changes which must be detected and resolved} \dots{} Dynamo is designed to be an eventually consistent data store; that is all updates reach all replicas eventually.
\end{quote}

The conflict detection and resolution mechanisms required by eventually consistent systems increase their complexity compared to consistent systems, which do not need them. This begs the question: are there techniques that could be used to improve availability without giving up on consistency, allowing for simpler systems? The \textsc{cap} theorem tells us that we can never build strictly consistent and maximally available systems, but a different trade-off might be possible.

\subsection{The Raft consensus algorithm\label{ssc:raft-consensus-algorithm}}

\subsection{Structured voting schemes\label{ssc:structured-voting-schemes}}

\section{Overview of Structured Rafter\label{sc:overview-of-structured-rafter}}

\chapter{Preparation\label{ch:preparation}}

% \begin{itemize}
    % \item Work undertaken before code was written
    % \item Refinement of project proposal
    % \item Professional approach: Requirements analysis!, reference Software Engineering techniques
    % \item PLs and systems which had to be learnt, theories and algorithms which required understanding
% \end{itemize}

\chapter{Implementation\label{ch:implementation}}

% \begin{itemize}
    % \item Describe what was actually produced
    % \item Design strategies (looking ahead to the testing stage)
    % \item Draw attention to the parts of the work which are not your own
    % \item Mention major milestones
% \end{itemize}

\chapter{Evaluation\label{ch:evaluation}}

% \begin{itemize}
    % \item Signs of success, evidence of thorough and systematic testing
    % \item Sample output, graphs, diagrams
    % \item Original goals achieved? Proof? Did stuff work?
% \end{itemize}

% \begin{itemize}
    % \item Failure modes presentation: most frequent failure modes are individual, then rack failures
\end{itemize}

\chapter{Conclusions\label{ch:conclusions}}

% \begin{itemize}
    % \item Refer to Introduction
    % \item Lessons learnt
% \end{itemize}

\printbibliography

\backmatter

\appendix

\chapter{Code samples\label{ch:code-samples}}

\end{document}
