\documentclass[10pt]{scrartcl}
\usepackage{fixltx2e} % LaTeX patches, \textsubscript
\usepackage{cmap} % fix search and cut-and-paste in Acrobat
\usepackage{ifthen}
\usepackage[oldstylenums,light,largesmallcaps]{kpfonts}
\usepackage[T1]{fontenc}
\usepackage[utf8]{inputenc}
\usepackage{tabularx}
\usepackage{amsmath}
\usepackage{mathtools}
\usepackage[colorlinks=true,linkcolor=blue,urlcolor=blue,citecolor=blue]{hyperref}

%%% Custom LaTeX preamble
% serif headings:
\addtokomafont{disposition}{\rmfamily}
\addtokomafont{descriptionlabel}{\rmfamily}
% no section numbering
\setcounter{secnumdepth}{0}
% hyperlinks
\urlstyle{same} % normal text font (alternatives: tt, rm, sf)
\hypersetup{
  pdftitle={Extending Raft with Structured Voting},
  pdfauthor={Leonhard Markert (lm510), Emmanuel College}
}
\title{Grid protocol probability analysis}
\author{Leonhard Markert (lm510), Emmanuel College}

%%% Body
\begin{document}

\maketitle

\section{Grid protocol description%
  \label{grid-protocol-description}%
}

``The structured Grid Protocol [\dots] arranges processes in a logical rectangular \(k \times j\) grid having \(k\) columns and \(j\) rows for a system comprising \(k \cdot j = n\) processes. [A] write quorum consists of all processes from a complete column plus one process from each column to meet the quorum system intersection property.'' \cite{voting}

\section{Quorum probability model%
  \label{quorum-probability-model}%
}

Given \(k \in \{0, 1, \dots, n\}\) votes, what is the probability \(\mathbb{P}(Q^{c,r}\;|\;V=k) = \mathbb{P}(Q^{c,r}_k)\) that the set of votes forms a quorum?

We assume here that the grid we are considering has \(r\) rows and \(c\) columns (let \(n = r c\) for convenience), therefore a complete column cover (\textsc{cc}-cover) consists of \(r\) nodes and a column cover (\textsc{c}-cover) consists of \(c\) nodes.

Let \(\mathbb{P}(CC^{c,r}\;|\;V=k)\ = \mathbb{P}(CC^{c,r}_k)\) and \(\mathbb{P}(C^{c,r}\;|\;V=k)\ = \mathbb{P}(C^{c,r}_k)\) be the probability of getting a \textsc{cc}-cover and a \textsc{c}-cover, respectively. Then \(\mathbb{P}(Q^{c,r}_k) = \mathbb{P}(CC^{c,r}_k\,\cap\,C^{c,r}_k) = \mathbb{P}(C^{c,r}_k\;|\;CC^{c,r}_k)\;\mathbb{P}(CC^{c,r}_k)\).

\subsection{\textsc{c}-cover probability}

In this section and the next, I will assume that all the binomials that occur are defined, which is not necessarily the case. This issue will be taken care of when we put everything together.

The number of ways in which we can get a \textsc{c}-cover equals the number of ways in which \(c\) columns can be covered by \(k\) nodes. We divide by the total number of ways in which \(k\) nodes can be distributed over \(n\) slots to get \[\mathbb{P}(C^{c,r}_k) = \dbinom{k}{c} \div \dbinom{n}{k}\]

Conditioning on the existence of a \textsc{cc}-cover, we find that there are \(c\) ways in which a \textsc{cc}-cover can come about, and so \(\mathbb{P}(C^{c,r}_k\;|\;CC^{c,r}_k) = c \, \mathbb{P}(C^{c-1,r}_{k-r}) \).

\subsection{\textsc{cc}-cover probability}

Consider how a \textsc{cc}-cover could be built. The first node could go anywhere in the grid. The second node would have to go into the same column, an event with probability \(\frac{r-1}{n-1}\). By similar reasoning, the third node would go into the same column with probability \(\frac{r-2}{n-2}\), and so forth. There are \(\binom{k}{r}\) different ways in which the \(k\) nodes at our disposal could cover those \(r\) slots. This gives

\begin{align*}
    \mathbb{P}(CC^{c,r}_k) &= \frac{r c}{n} \cdot \frac{r-1}{n-1} \cdot \frac{r-2}{n-2} \cdot \; \cdots \; \cdot \frac{r-(r-1)}{n-(r-1)} \cdot \dbinom{k}{r} \\
    &= c \cdot \frac{r!}{n! \div (n-r)!} \cdot \dbinom{k}{r} \\
    &= c \cdot \dbinom{k}{r} \div \dbinom{n}{r}
\end{align*}

\subsection{Quorum probability}

Putting everything so far together, we have

\begin{align*}
    \mathbb{P}(Q^{c,r}_k)
    &= \mathbb{P}(C^{c,r}_k\;|\;CC^{c,r}_k) \; \mathbb{P}(CC^{c,r}_k) \\
    &= c \, \mathbb{P}(C^{c-1,r}_{k-r}) \; \mathbb{P}(CC^{c,r}_k) \\
    &= c \, \left\{\dbinom{k-r}{c-1} \div \dbinom{n-r}{k-r}\right\} \, \left\{c \, \dbinom{k}{r} \div \dbinom{n}{r}\right\} \\
    &= c^2 \, \frac{\dbinom{k-r}{c-1} \dbinom{k}{r}}{\dbinom{n-r}{k-r} \dbinom{n}{r}}
\end{align*}

By the definition of the binomial, this is only defined if \(n \geq k\) (which is true by assumption), \(k \geq r+c-1\) and \(n \geq r+c-1\) (which is implied by transitivity). We know that \(r+c-1\) nodes are needed in order to get a \textsc{c}-cover as well as a \textsc{cc}-cover, so if \(k < r+c-1\) there can be no quorum.

The second special case is when we are sure to get a quorum. Starting from a full grid, we can take \(r-1\) nodes away and still get a  \textsc{c}-cover, or we can take \(c-1\) nodes away and still get a \textsc{cc}-cover. Thus if \(n-k < \min(r,c)\) we will always find a quorum.

This gives us the final expression for the probability of finding a quorum given \(k\) out of \(n\) possible votes,

\[
\mathbb{P}(Q^{c,r}_k) =
    \begin{cases}
        0 & \text{if}\;k < r+c-1 \\
        1 & \text{if}\;k > n - \min(r,c) \\
        {c \cdot \begin{pmatrix}n-(r+c-1) \\ k-(r+c-1)\end{pmatrix}} \div {\begin{pmatrix}n \\ k\end{pmatrix}} & \text{otherwise}
    \end{cases}
\]

\bibliographystyle{amsplain}
\bibliography{Bibliography}

\end{document}
