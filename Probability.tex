\documentclass[10pt]{scrartcl}
\usepackage{fixltx2e} % LaTeX patches, \textsubscript
\usepackage{cmap} % fix search and cut-and-paste in Acrobat
\usepackage{ifthen}
\usepackage[oldstylenums,light,largesmallcaps]{kpfonts}
\usepackage[T1]{fontenc}
\usepackage[utf8]{inputenc}
\usepackage{tabularx}
\usepackage{amsmath}
\usepackage{mathtools}
\usepackage[colorlinks=true,linkcolor=blue,urlcolor=blue,citecolor=blue]{hyperref}

%%% Custom LaTeX preamble
% serif headings:
\addtokomafont{disposition}{\rmfamily}
\addtokomafont{descriptionlabel}{\rmfamily}
% no section numbering
\setcounter{secnumdepth}{0}
% hyperlinks
\urlstyle{same} % normal text font (alternatives: tt, rm, sf)
\hypersetup{
  pdftitle={Extending Raft with Structured Voting},
  pdfauthor={Leonhard Markert (lm510), Emmanuel College}
}
\title{Grid protocol probability analysis}
\author{Leonhard Markert (lm510), Emmanuel College}

%%% Body
\begin{document}

\maketitle

\section{Grid protocol description%
  \label{grid-protocol-description}%
}

``The structured Grid Protocol [\dots] arranges processes in a logical rectangular \(k \times j\) grid having \(k\) columns and \(j\) rows for a system comprising \(k \cdot j = n\) processes. [A] write quorum consists of all processes from a complete column plus one process from each column to meet the quorum system intersection property.'' \cite{voting}

\section{Quorum probability model%
  \label{quorum-probability-model}%
}

Our grid has \(r\) rows and \(c\) columns (let \(n = r c\) for convenience), therefore a complete column cover (\textsc{cc}-cover) consists of \(r\) nodes and a column cover (\textsc{c}-cover) consists of \(c\) nodes.

Given \(k \in \{0, 1, \dots, n\}\) votes, what is the probability \(\mathbb{P}(Q\;|\;V=k) = \mathbb{P}(Q_k)\) that the set of votes forms a quorum?

Let \(\mathbb{P}(CC\;|\;V=k)\ = \mathbb{P}(CC_k)\) and \(\mathbb{P}(C\;|\;V=k)\ = \mathbb{P}(C_k)\) be the probability of getting a \textsc{cc}-cover and a \textsc{c}-cover, respectively. Then \(\mathbb{P}(Q_k) = \mathbb{P}(CC_k\,\cap\,C_k) = \mathbb{P}(CC_k\;|\;C_k)\;\mathbb{P}(C_k)\).

\subsection{\textsc{c}-cover probability}

In this section and the next, I will assume that all the binomials we're using are defined, which is not necessarily the case. This issue will be taken care of when we put everything together.

The probability of getting a \textsc{c}-cover with \(k\) nodes is simply the number of ways in which the \(k-c\) nodes that are not part of the \textsc{c}-cover can distribute over the \(n-c\) elements of the grid not taken up by the \textsc{c}-cover, divided by the number of ways in which any \(k\) nodes can distribute over a grid with \(n\) spaces: \[\mathbb{P}(C_k) = {\begin{pmatrix}n-c \\ k-c\end{pmatrix}} \div {\begin{pmatrix}n \\ k\end{pmatrix}}\]

\subsection{\textsc{cc}-cover probability}

Given that we have a \textsc{c}-cover, there are \[\binom{n-c}{k-c}\] different ways are there in which the \(k - c \) nodes that are not part of the \textsc{c}-cover can be distributed over the grid.

Again given that there is a \textsc{c}-cover, in how many different ways could there be a \textsc{cc}-cover? Firstly, the \textsc{cc}-cover could cover any of the \(c\) columns. Now taking the \(r+c-1\) nodes in the \textsc{c}-cover and those in the \textsc{cc}-cover as fixed, there are \[\binom{n-(r+c-1)}{k-(r+c-1)}\] ways in which the remaining nodes can distribute over the grid.

Thus the probability of a \textsc{cc}-cover given a \textsc{c}-cover is \[\mathbb{P}(CC_k\;|\;C_k) = {c \cdot \begin{pmatrix}n-(r+c-1) \\ k-(r+c-1)\end{pmatrix}} \div {\begin{pmatrix}n-c \\ k-c\end{pmatrix}}\].

\subsection{Quorum probability}

Putting everything so far together, we have

\begin{align*}
\mathbb{P}(Q_k) &= \mathbb{P}(CC_k\;|\;C_k)\;\mathbb{P}(C_k) \\
                &= \left\{{c \cdot \begin{pmatrix}n-(r+c-1) \\ k-(r+c-1)\end{pmatrix}} \div {\begin{pmatrix}n-c \\ k-c\end{pmatrix}}\right\} \cdot \left\{{\begin{pmatrix}n-c \\ k-c\end{pmatrix}} \div {\begin{pmatrix}n \\ k\end{pmatrix}}\right\} \\
                &= {c \cdot \begin{pmatrix}n-(r+c-1) \\ k-(r+c-1)\end{pmatrix}} \div {\begin{pmatrix}n \\ k\end{pmatrix}}
\end{align*}

By the definition of the binomial, this is only defined if \(n \geq k\) (which is true by assumption), \(k \geq r+c-1\) and \(n \geq r+c-1\) (which is implied by transitivity). We know that \(r+c-1\) nodes are needed in order to get a \textsc{c}-cover as well as a \textsc{cc}-cover, so if \(k < r+c-1\) there can be no quorum.

The second special case is when we are sure to get a quorum. Starting from a full grid, we can take \(r-1\) nodes away and still get a  \textsc{c}-cover, or we can take \(c-1\) nodes away and still get a \textsc{cc}-cover. Thus if \(n-k < \min(r,c)\) we will always find a quorum.

This gives us the final expression for the probability of finding a quorum given \(k\) out of \(n\) possible votes,

\[
\mathbb{P}(Q_k) =
    \begin{cases}
        0 & \text{if}\;k < r+c-1 \\
        1 & \text{if}\;k > n - \min(r,c) \\
        {c \cdot \begin{pmatrix}n-(r+c-1) \\ k-(r+c-1)\end{pmatrix}} \div {\begin{pmatrix}n \\ k\end{pmatrix}} & \text{otherwise}
    \end{cases}
\]

\bibliographystyle{amsplain}
\bibliography{Bibliography}

\end{document}
