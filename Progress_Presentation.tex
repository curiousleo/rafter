\documentclass{beamer}
\usepackage{fixltx2e} % LaTeX patches, \textsubscript
\usepackage{cmap} % fix search and cut-and-paste in Acrobat
\usepackage[oldstylenums,largesmallcaps]{kpfonts}
\usepackage[T1]{fontenc}
\usepackage[utf8]{inputenc}

\usetheme{Pittsburgh}
\usefonttheme{serif}
\title{Extending Raft with Structured Voting}
\subtitle{Progress presentation}
\author{Leonhard Markert}
% \institute{Emmanuel College}
\date{\today}

%% Custom commands
\newcommand{\voted}[1]{\textbf{\usebeamercolor[fg]{frametitle}#1}}

\begin{document}

%% Prelude

\begin{frame}
    \titlepage
\end{frame}

% \section*{Outline}
% \begin{frame}
    % \tableofcontents
% \end{frame}

%% Main content

\section{Introduction}

% Raft is a new consensus algorithm to be used instead of Paxos, aiming to be easier to understand and implement correctly. It models the distributed system as a replicated state machine using a replicated log. Raft features a strong leader, an explicit membership change mechanism and a (partial) correctness proof.

% Any two quorums overlap.

% Structured voting schemes impose a logical structure on the set of processes and use structural properties to specify quorum systems. Each voting scheme defines a way to construct write quorums, any two of which overlap.

\begin{frame}{Introduction}
    \begin{columns}[t]
        \begin{column}{.5\textwidth}
            \begin{block}{Raft consensus algorithm}
                \begin{itemize}
                    \item Strongly consistent
                    \item Uses majority voting
                    % \item Replicated state machine model
                    \item Partial correctness proof
                    \item Requirement: \emph{Any two quorums overlap.}
                    % \item Rafter implements Raft
                \end{itemize}
            \end{block}
        \end{column}
        \begin{column}{.5\textwidth}
            \begin{block}{Structured voting}
                \begin{itemize}
                    \item Impose logical structure
                    \item Result: smaller quorums
                \end{itemize}

                \begin{exampleblock}{Example: Grid protocol}
                    \medskip
                    \begin{columns}
                        \begin{column}{.2\textwidth}
                            \begin{tabular}{c | c | c}
                                A & \voted{B} & C \\
                                \hline
                                D & \voted{E} & \voted{F} \\
                                \hline
                                \voted{G} & \voted{H} & I \\
                                \hline
                                J & \voted{K} & L \\
                            \end{tabular}
                        \end{column}
                        \begin{column}{.40\textwidth}
                            \small{One node from each column and one complete column}
                        \end{column}
                    \end{columns}
                \end{exampleblock}
            \end{block}
        \end{column}
    \end{columns}
\end{frame}

\section{Progress}

\section{Challenges}

\section{To do}

%% Afterlude

\begin{frame}
    % enforce entries in the ToC
\end{frame}

\end{document}
