\documentclass[11pt]{scrartcl}
\usepackage{fixltx2e} % LaTeX patches, \textsubscript
\usepackage{cmap} % fix search and cut-and-paste in Acrobat
\usepackage{ifthen}
\usepackage[oldstylenums,light,largesmallcaps]{kpfonts}
\usepackage[T1]{fontenc}
\usepackage[utf8]{inputenc}
\usepackage{tabularx}
\usepackage[colorlinks=true,linkcolor=blue,urlcolor=blue,citecolor=blue]{hyperref}

%%% Custom LaTeX preamble
% serif headings:
\addtokomafont{disposition}{\rmfamily}
\addtokomafont{descriptionlabel}{\rmfamily}
% no section numbering
\setcounter{secnumdepth}{0}
% hyperlinks
\urlstyle{same} % normal text font (alternatives: tt, rm, sf)
\hypersetup{
  pdftitle={Extending Raft with Structured Voting},
  pdfauthor={Leonhard Markert (lm510), Emmanuel College}
}

%%% Body
\begin{document}
\thispagestyle{empty}

\rightline{\large\it Leonhard Markert} \medskip
\rightline{\large\it Emmanuel College} \medskip
\rightline{\large\it lm510}

\vfil

\centerline{\large Computer Science Part \textsc{ii} Progress Report} \vspace{0.4in}
\centerline{\Large\bf Extending Raft with Structured Voting} \vspace{0.3in}
\centerline{\large\it \today}

\vfil

\begin{center}
\begin{tabularx}{316pt}{rX}
\textbf{Project Supervisors} & \textit{Malte Schwarzkopf} and \textit{Ionel Gog} \\ \\
\textbf{Director of Studies} & \textit{Dr Jonathan Hayman} \\ \\
\textbf{Overseers} & \textit{Dr Markus Kuhn} and \textit{Dr Neal Lathia}
\end{tabularx}
\end{center}

\newpage

\section{Progress Report%
  \label{progress-report}%
}

\textbf{The project is on schedule and all milestones so far have been reached.} During Michaelmas term, I fell behind by three weeks, but fortunately after lectures and supervisions had finished, I was able to catch up with the work plan and have stayed on top of it since.

The project work up to now took place in four phases: I started by familiarising myself with the programming language Erlang, then developed the core voting algorithms, integrated my code with Rafter\footnote{Rafter is an Erlang library by Andrew J.\,Stone which implements the Raft consensus protocol. \url{https://github.com/andrewjstone/rafter}.}, and finally implementated a Memcached\footnote{Memcached is a popular memory object caching system. \url{http://memcached.org}.} compatibility layer.

\begin{description}
    \item[Familiarisation with Erlang.] I started the project by implementing a purely functional Deque following \cite{okasaki} with type annotations for Dialyzer\footnote{Dialyzer is a tool used to statically type check Erlang programs. \url{http://www.erlang.org/doc/man/dialyzer.html}.} and QuickCheck\footnote{QuickCheck is a commercial property-based Erlang testing framework by Quviq \textsc{ab}. \url{http://www.quiviq.com}.} unit tests. In order to learn about asynchronous programming with Erlang, I wrote a simplistic distributed benchmarking tool. Writing idiomatic asynchronous code in Erlang turned out to be more difficult than I had anticipated, and this was why I fell behind the schedule during Michaelmas term.
    \item[Development of core algorithms.] About three weeks were spent designing voting structure and voting state type specifications, writing a visualisation tool for both, and implementing voting structure generators for the Majority Protocol and the Grid Protocol.
    \item[Integration with Rafter.] Due to the modular design of Rafter, integrating my structured voting code took only a weekend. Adapting the test suite to cover the new code took about two weeks.
    \item[Memcached compatibility layer.] After some deliberation, I decided to implement only the \texttt{set} and \texttt{get} operations using Memcached's binary protocol. This degree of compatibility is sufficient to run Memcached benchmarks and obtain meaningful results whithout too much effort.
\end{description}

\bibliographystyle{amsplain}
\bibliography{Bibliography}

\end{document}
